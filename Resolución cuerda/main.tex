\documentclass{article}
\usepackage[utf8]{inputenc}
\usepackage[utf8]{inputenc}
\usepackage{amsmath}
\usepackage{amssymb}
\usepackage{commath}
\usepackage{cancel}
\usepackage{graphicx}
\usepackage{comment}
\usepackage{cases}
\usepackage{xcolor}
\usepackage{hyperref}

\title{Resolución cuerda}
\author{Rodrigo Tesone}
\date{April 2021}

\begin{document}

\maketitle

\section*{Resolución de ecuación de una cuerda}
\subsection*{2)}
La idea es resolver:
\begin{equation}\label{Derivada_parcial_cuerda}
    \frac{\partial^2 u}{\partial x^2}=\frac{1}{c^2}\frac{\partial^2 u}{\partial t^2}
\end{equation}
Con las condiciones de contorno: \\
\begin{numcases}{}
    u(0,t)=0\label{Condicion_contorno_1}\\
    u(L,t)=0\label{Condicion_contorno_2}
\end{numcases}
Y las condiciones iniciales:
\begin{equation}\label{Condicion_incial_1}
    u(x,0) = f(x)=
    \begin{cases}
        \frac{2c}{L}x & \text{for } 0\leq x \leq \frac{L}{2}\\[10px]
        \frac{2c}{L}(L-x) & \text{for } \frac{L}{2} < x \leq L \\
    \end{cases}
\end{equation}
\begin{equation}\label{Condicion_inicial_2}
    \frac{\partial u}{\partial t} (x,0)= 0
\end{equation}
Planteo:$$u(x,t)=A(x)B(t)$$
Calculo las derivadas:
\begin{equation*}
    \begin{split}
        \frac{\partial^2 u}{\partial x^2}=A^{''}(x)B(t)\\
        \frac{\partial^2 u}{\partial t^2}=A(x)B^{''}(t)\\
    \end{split}
\end{equation*}
Reemplazo en \eqref{Derivada_parcial_cuerda} y trabajo algebraicamente:
\begin{equation*}
    \begin{split}
        A^{''}(x)B(t)&=\frac{1}{c^2}A(x)B^{''}(t)\\
        \frac{A^{''}(x)}{A(x)} & = \frac{1}{c^2}\frac{B^{''}(t)}{B(t)}
    \end{split}
\end{equation*}
Dado que se esta igualando entre 2 funciones de variables diferentes la única solución posible es que la razón sobre esas funciones sea una constante llamada $-\lambda$.\\
Considerando eso planteamos las siguientes ecuaciones diferenciales:
\begin{numcases}{}
    A^{''}(x)+\lambda A(x)&=0 \label{eq:x} \\
    B^{''}(t)+\lambda c^2B(t)&=0 \label{eq:t}
\end{numcases}
De aqui vemos que se reduce a 2 ecuaciones diferenciales de una sola variable.\\
La solución de \eqref{eq:x} es:
\begin{equation*}
    A(x) =C_1 \sin\left({\sqrt{\lambda}x}\right)+C_2\cos\left({\sqrt{\lambda}x}\right)
\end{equation*}
Le aplico la condición \eqref{Condicion_contorno_1}:
\begin{equation*}
    \begin{split}
        u(0,t)=A(0)B(t)&=0\\
        A(0)&=0\\
        C_2\cos(0)&=0\\
        C_2 &=0
    \end{split}
\end{equation*}
Ahora aplico \eqref{Condicion_contorno_2}:
\begin{equation*}
    \begin{split}
        u(L,t)=A(L)B(t)&=0\\
        A(L)&=0\\
        C_1 \sin\left({\sqrt{\lambda}L}\right) &=0\\
        \text{Dado que $C_1$ no puede ser 0}\\
        \sqrt{\lambda}L &= n\pi\\
        \sqrt{\lambda} &=\frac{n\pi}{L}
    \end{split}
\end{equation*}
Con esta relación para $\lambda$ resolvemos \eqref{eq:t} proponiendo:
\begin{equation*}
    B(t) = D_1 \sin\left({\sqrt{\lambda}x}\right)+D_2\cos\left({\sqrt{\lambda}x}\right)
\end{equation*}
Le aplico la condición \eqref{Condicion_inicial_2}:
\begin{equation*}
    \begin{split}
        \frac{\partial u}{\partial t} &= \frac{\partial (A(x)B(t))}{\partial t}=A(x)\frac{\partial B(t)}{\partial t}=A(x)\left[ D_1\sqrt{\lambda} \cos\left({\sqrt{\lambda}x}\right)-D_2\sqrt{\lambda}\sin\left({\sqrt{\lambda}x}\right) \right]\\
        \frac{\partial u}{\partial t} (x,0)&= A(x)\left[ D_1\sqrt{\lambda} \cos\left({\sqrt{\lambda}0}\right)-D_2\sqrt{\lambda}\sin\left({\sqrt{\lambda}0}\right) \right]\\
        \frac{\partial u}{\partial t} (x,0)&=A(x)D_1\sqrt{\lambda}\\
        0 &=A(x)D_1\sqrt{\lambda}\\
        D_1 &=0  
        \end{split}
\end{equation*}
$\therefore$ 
\begin{equation*}
    u(x,t)=\sum_{n=1}^{+\infty}c_n\sin{\left(\frac{n\pi}{L}x\right)}\cos{\left(\frac{n\pi}{L}ct\right)}
\end{equation*}
Para plantear la condición inicial \eqref{Condicion_incial_1} primero reescribimos a $f(x)$ como $\widetilde{f}(x)$ siendo $\widetilde{f}(x)$ una función impar:
\begin{equation*}
    \widetilde{f}(x)=
    \begin{cases}
    -\frac{2c}{L}(L+x) & \text{for } -L < x \leq -\frac{L}{2} \\[10px]
    \frac{2c}{L}x & \text{for } -\frac{l}{2}\leq x \leq 0\\[10px]
    \frac{2c}{L}x & \text{for } 0\leq x \leq \frac{L}{2}\\[10px]
    \frac{2c}{L}(L-x) & \text{for } \frac{L}{2} < x \leq L \\
    \end{cases}
\end{equation*}
Ahora calculamos 
\begin{equation*}
    \widetilde{f}(x)=\sum_{n=1}^{+\infty}c_n\sin{\left(\frac{n\pi}{L}x\right)}
\end{equation*}
Solo falta obtener $c_n$ que es el termino que resulta de calcular una serie de Fourier de senos:
\begin{equation*}
    \begin{split}
        c_n &= \frac{2}{L}\int_{0}^L f(x)\sin{\left(\frac{n\pi}{L}x\right)} dx \\
        c_n &= \frac{2}{L}\left[\underbrace{\int_0^{\frac{L}{2}}\frac{2c}{L}x\sin{\left(\frac{n\pi}{L}x\right)}}_{f_1}+\underbrace{\int_{\frac{L}{2}}^{L} \frac{2c}{L}(L-x)\sin{\left(\frac{n\pi}{L}x\right)}dx}_{f_2} \right]
    \end{split}
\end{equation*}
Luego de resolverlas en el jupyternotbook vemos:
\begin{equation*}
    c_n = \frac{2}{L}\left[ \frac{2 C \left(- \frac{L^{2} \cos{\left(\frac{n \pi}{2} \right)}}{2 n \pi} + \frac{L^{2} \sin{\left(\frac{n \pi}{2} \right)}}{n^{2} \pi^{2}}\right)}{L}-    \frac{2 C L \sin{\left(n \pi \right)}}{n^{2} \pi^{2}} - \frac{2 C \left(- \frac{L^{2} \cos{\left(\frac{n \pi}{2} \right)}}{2 n \pi} - \frac{L^{2} \sin{\left(\frac{n \pi}{2} \right)}}{n^{2} \pi^{2}}\right)}{L} \right]
\end{equation*}
Reacomodando terminos se ve:
\begin{equation*}
    c_n = \frac{8c}{n^2\pi^2}\sin{\left(n\frac{\pi}{2}\right)}
\end{equation*}
Los pares se anulan asique pruebo con $n=2n-1$:
\begin{equation*}
    c_{2n-1} = \frac{8c}{(2n-1)^2\pi^2}(-1)^{n+1}
\end{equation*}
$\therefore$ 
\begin{equation*}
    u(x,t)=\sum_{n=1}^{+\infty}\frac{8c}{(2n-1)^2\pi^2}(-1)^{n+1}\sin{\left(\frac{(2n-1)\pi}{L}x\right)}\cos{\left(\frac{(2n-1)\pi}{L}ct\right)}
\end{equation*}
\end{document}
